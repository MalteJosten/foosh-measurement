% spellchecker:disable
\documentclass[pdftex,
	chapterprefix,
	headsepline,
	footsepline,
	% colordvi,
	11pt,
	a4paper,
	parskip=half,
	% enabledeprecatedfontcommands,
	final,
	appendixprefix,
	bibliography=totoc]{scrbook}

% define a new if command to distinguish between PDF and DVI output
\usepackage{ifpdf}
\ifx\pdfoutput\undefined
\pdffalse %not PDFLaTeX
\else
\pdfoutput=1
\pdftrue
\fi

% language support (hyphenation etc)
%% DO NOT CHANGE THIS!
% Select the document language with the \selectlanguage command below.
\usepackage[german,english]{babel}

% for prettier tables
\usepackage{booktabs}

% support for latin1 characters. That means you can enter umlauts directly
% no need for "a "u "o "s anymore
\usepackage[utf8]{inputenc}
\usepackage{textcomp}

% provides the \url{} command to pretty print urls
\usepackage{url}

% allows to \includegraphics
\usepackage{graphicx}
\usepackage{svg}

% defines some standard colornames like "black" etc.
\usepackage{color}

% allows to color tablecells
\usepackage{colortbl}

% provides an easier interface to if-then-else constructs in custom macros
\usepackage{ifthen}

% allows tables to break over pages.
\usepackage{supertabular}

% allows to have different kinds paper orientations in the same pdf-documnent
\usepackage{pdflscape}

% allows to specify absolute texpos for textboxes. This is generally only important for the titlepage
\usepackage[absolute]{textpos}

% % allows to enumerate different figures with a) b) in the same figure-environment.
\usepackage{subfigure}

\usepackage{caption}
\DeclareCaptionLabelFormat{cont}{#1~#2\alph{ContinuedFloat}}
\captionsetup[ContinuedFloat]{labelformat=cont}

% math
\usepackage{amsmath}

% allows inserting source code
\usepackage{listings}
\usepackage{scrhack} % fixes lst incompatibility
\lstset{
% numbers=left,              % Location of line numbers
	numberstyle=\tiny,          % Style of line numbers
	numbersep=5pt,              % Margin between line numbers and text
	tabsize=4,                  % Size of tabs
	extendedchars=true,
	breaklines=true,            % Lines will be wrapped
	showspaces=false,
	showtabs=false,
	showstringspaces=false
}

% More fancy syntax highlighting: https://texdoc.net/texmf-dist/doc/latex/minted/minted.pdf
% careful, this has additional external dependencies
%\usepackage[newfloat]{minted}
% configure source code highlighting; edit this to your liking
% style=bw will save you some money on color pages, but may be harder to read
%\setminted{
%	style=stata-light,
%	frame=leftline,
%	breaklines=true,
%	numbers=left,
%	tabsize=4}

% allows usage of BibLaTeX for bibliography; needs the Biber tool!
\usepackage[
backend=biber,
sorting=anyt
]{biblatex}
\usepackage{csquotes}
% point this to your bibliography file
\addbibresource{bib/references.bib}

% finetune the gaps between figure and text in the subfigure environment (basically close the gap as much as possible)
\renewcommand{\subfigtopskip}{0pt}
\renewcommand{\subfigbottomskip}{0pt}

% some color definitions for the pdf statements below
\definecolor{mygrey}{rgb}{0.45,0.45,0.45}
\definecolor{mydarkgrey}{rgb}{0.2,0.2,0.2}
\definecolor{red}{rgb}{1.0,0.33,0.33}
\definecolor{orange}{rgb}{1.00,0.73,0.33}
\definecolor{yellow}{rgb}{0.95,0.92,0.}
\definecolor{lightgreen}{rgb}{0.3,0.95,0.46}
\definecolor{titleblue}{rgb}{0.03,0.10,0.46}

\ifpdf
% For screen viewing it is nice to have references marked in a slightly different
% color than the rest of the text. Since they will be hyperlinks to the
% referenced objects.
\usepackage[pdftex,
	pdftitle={Fancy thesis template},
	% colorlinks,
	% linkcolor={black},
	% citecolor={black},
	urlcolor={black},
	plainpages={false},
	bookmarksnumbered={true},
	pdfauthor={Vorname Nachname},
	pdfsubject={},
	pdfkeywords={},
	pdfstartview={FitBH}]{hyperref}

\pdfcompresslevel=9
\fi

% Removes dots at the end of chapter headings when appendix is used.
\KOMAoptions{numbers=noendperiod}

% some configuration for the amount of text on a single page
\usepackage{typearea}
\areaset[1.5cm]{418pt}{658pt}
\setlength{\headheight}{37pt}

% enter author and title for the titlepage.
\author{}
\title{}

% To avoid nasty mistakes like having comments directly in the textflow
% the following \todo macro was defined. With that you can enter
% \todo{What I still have to do here}
% inside of your text and a marker will appear at the page's margin with the
% text "What I still have to do here".
% The first line activates this feature. If you comment it out and uncomment
% the second line below there will be no error messages and no todos will be shown
% anymore. So - even if you have forgotten to delete one of them - they will not appear
% in the final printout.
\newcommand{\todo}[1]{\marginpar{\textcolor{red}{TODO:} #1}}

% We recommend to split your document into several files. Usually one for every chapter is a
% good idea. If you follow this guideline (how to assemble these files in a single document
% see two paragraphs below) you will be able to single out chapters via the \includeonly{}
% command. Using this mechanism page numbering and references of the full run before will be
% preserved. This also nice, if your latex run tends to get slow and you need to fine tune
% some formatting in one chapter - just include that one. The rest (or at least the ones before
% the one currently under construction) will remain untouched. This means a boost in compilation time.
%\includeonly{content/chapter2}

\begin{document}

% the next two lines influence the detailedness of the table of contents
% and to what structure depth numbers are written before sections/subsections/paragraphs
% You should not touch this
\setcounter{tocdepth}{2}
\setcounter{secnumdepth}{3}

\frontmatter

% here the titlepage is included. Look into the file "frontpage.tex" to
% adapt it to your needs (name, title etc.)
\def\front_lang{1}

\if\front_lang0
	% spellchecker:disable
\begin{titlepage}
\selectlanguage{german}
\vspace*{-1cm}
\newlength{\links}
\setlength{\links}{0.9cm}
\setlength{\TPHorizModule}{1cm}
\setlength{\TPVertModule}{1cm}
\textblockorigin{0pt}{0pt}

\sffamily
\LARGE

\begin{textblock}{16.5}(2.8,2.6)
 \hspace*{-0.25cm} \textbf{UNIVERSITÄT DUISBURG-ESSEN} \\
 \hspace*{-1.15cm} \rule{5mm}{5mm} \hspace*{0.05cm} FAKULTÄT FÜR INGENIEURWISSENSCHAFTEN\\
 \large{}ABTEILUNG INFORMATIK UND ANGEWANDTE KOGNITIONSWISSENSCHAFT\\
\end{textblock}


%Hier Titel, Name, und Matrikelnummer eintragen, \\ make a newline
\begin{textblock}{14.5}(3.2,8.5)
  \large
{ \textbf{Masterarbeit}} \\[1cm]
{\LARGE \Large\textbf{FooSH: A Framework for outcome-oriented Smart Homes}} \\[1.3cm]
Malte Josten\\
Matrikelnummer: 3066184\\
Angewandte Informatik (Master)
\end{textblock}



\begin{textblock}{10}(10.5,17.5)
\includegraphics[scale=1.0]{images/unilogo.pdf}\\
\normalsize
\raggedleft
Fachgebiet Verteilte Systeme, Abteilung Informatik \\
Fakultät für Ingenieurwissenschaften \\
Universität Duisburg-Essen \\[2ex]

\today\\[15ex]
\raggedright
% Supervisors
{\textbf{Erstgutachter:}} Prof. Dr-Ing. Torben Weis \\
{\textbf{Zweitgutachter:}} Prof. Dr. Gregor Schiele \\
{\textbf{Zeitraum:}} 5. Juni 2023 - 4. Dezember 2023 \\
\end{textblock}

\end{titlepage}

\else
	% spellchecker:disable
\begin{titlepage}
\selectlanguage{english}
\vspace*{-1cm}
\newlength{\links}
\setlength{\links}{0.9cm}
\setlength{\TPHorizModule}{1cm}
\setlength{\TPVertModule}{1cm}
\textblockorigin{0pt}{0pt}

\sffamily
\LARGE

\begin{textblock}{16.5}(2.8,2.6)
 \hspace*{-0.25cm} \textbf{UNIVERSITY OF DUISBURG-ESSEN} \\
 \hspace*{-1.15cm} \rule{5mm}{5mm} \hspace*{0.05cm} FACULTY OF ENGINEERING \\
 \large{}DEPARTMENT OF COMPUTER SCIENCE AND APPLIED COGNITIVE SCIENCE\\
\end{textblock}


%Hier Titel, Name, und Matrikelnummer eintragen, \\ make a newline
\begin{textblock}{14.5}(3.2,8.5)
  \large
{ \textbf{Master's thesis}} \\[1cm]
{\LARGE \Large\textbf{FooSH: A Framework for outcome-oriented Smart Homes}} \\[1.3cm]
Malte Josten\\
Matriculation number: 3066184\\
Applied Computer Science (Master)
\end{textblock}



\begin{textblock}{10}(10.5,17.5)
\includegraphics[scale=1.0]{images/unilogo.pdf}\\
\normalsize
\raggedleft
Distributed Systems Group, Department of Computer Science \\
Faculty of Engineering \\
University of Duisburg-Essen \\[2ex]

\today\\[15ex]
\raggedright
% Supervisors
{\textbf{First Reviewer:}} Prof. Dr-Ing. Torben Weis \\
{\textbf{Second Reviewer:}} Prof. Dr. Gregor Schiele \\
{\textbf{Period of time:}} June 5 2023 - December 4 2023 \\
\end{textblock}

\end{titlepage}

\fi

% Select either german or english here
\selectlanguage{english}

\chapter*{Abstract}

An abstract is a brief summary of a research article, thesis, review, conference proceeding,
or any in-depth analysis of a particular subject
and is often used to help the reader quickly ascertain the paper's purpose.
When used, an abstract always appears at the beginning of a manuscript or typescript,
acting as the point-of-entry for any given academic paper or patent application.
Abstracting and indexing services for various academic disciplines
are aimed at compiling a body of literature for that particular subject.
\footnote{Wikipedia: \url{https://en.wikipedia.org/wiki/Abstract_(summary)}}



\tableofcontents

%\listoffigures
\mainmatter

% To assemble the whole document
% Please be aware that each file will begin on a new page
% therefore chapters should be put into such a file.
% There cannot be an include statement inside of an "included" file.
% So if you want to further divide your document - use \input inside of
% the included files. \input will not begin on a new page.
\chapter{Introduction}

The introduction is a short overview of the context and goals of the paper.
It describes the structure and contents of the paper
to provide the reader with enough information
to find parts of relevance to their current interest quickly.

If you are unfamiliar with {\LaTeX},
you will probably need to take some time to learn the general workflow.
However, this will save you a lot of time and headaches later on,
as it allows you to completely focus on the content
instead of managing layouts and correct citation.
There are multiple great guides and introductions on using it.~\cite{overleaf,learnlatex,wiki}

You can also find a searchable and browsable list of all common {\LaTeX} documentation online.~\cite{texdoc}

\chapter{Background}\label{ch:background}

\section{Smart Home Technologies}\label{ch:background-smart_home}
\section{(Software) Framework Principles}\label{ch:background-principles}
\section{Prediction Techniques}\label{ch:background-prediction_techniques}
\chapter{Related Work}\label{ch:related_work}

This chapter presents the basis on which the system and later evaluation are built.
It usually contains short descriptions of previous research papers
and puts them into the context of the thesis.

\section{Code Listings}

If there is some interesting code you would like to show
in order to ease the understanding of the text,
you can just include it using the \verb+lstlisting+ environment.
Have a look at the source of this page to see how this is included:

\begin{lstlisting}[language=Go]
x := from(42);
\end{lstlisting}

You could also put the code into an external file
and include it in this document using the \verb+lstinputlisting+ command:

\lstinputlisting[language=Go,numbers=left]{listings/example.go}

Be careful not to include large files as it hampers readability.
If there is a short excerpt from a large file you would like to show,
you can also extract an explicit range of lines from it without the need to modify the source file.
This next listing only shows the conditional from the previous code:

\lstinputlisting[language=Go,numbers=left,firstline=2,lastline=5,firstnumber=2]{listings/example.go}

To use more advanced syntax highlighting
have a look at the available options of the \emph{listings} package
or use the \emph{minted} package\footnote{\url{https://texdoc.net/texmf-dist/doc/latex/minted/minted.pdf}},
which has more extensive language support and additional themes.
Both can be configured in the main file.
 % input does not start a new page
\section{Math}

In case you need to include some math,
the \emph{amsmath} package\footnote{\url{https://texdoc.net/texmf-dist/doc/latex/amsmath/amsmath.pdf}} is already included in this document.

To properly display some short formula like \( e^{i \pi} = -1\),
you can use the \verb+\( \)+ inline command.
For larger formulas, the \verb+math+ environment is more appropriate.
If you need to reference the formula multiple times,
e.g. in case it is used in theorems,
you should use the \verb+equation+ environment:

\begin{equation}
	\vec\nabla\times\vec{B}= \mu_0\vec{j}+\mu_0\varepsilon_0\frac{\partial\vec{E}}{\partial t}
	\label{func}
\end{equation}

To reference it as~\ref{func} using the \verb+\ref{}+ command,
remember to use a \verb+\label{}+.

\section{Miscellaneous}

You can use the \verb+\todo{}+ command to put obvious reminders on the side of the document.\todo{like this!}


\chapter{Framework}\label{ch:framework}

\section{Design \& Architecture}\label{ch:framework-design_and_architecture}
\section{Implementation}\label{ch:framework-implementation}
\chapter{Proof of Concept}\label{ch:proof_of_concept}

\section{Available Data}\label{ch:poc-available_data}
\section{Artificial Intelligence}\label{ch:poc-artifical_intelligence}
\section{Symbolic Regression}\label{ch:poc-symbolic_regression}
\section{Validation}\label{ch:poc-validation}
\chapter{Evaluation}\label{ch:eval}

\chapter{Conclusion}

The conclusion quickly summarizes the results of the paper
in relation to the previously defined goals and hypotheses.
It usually also includes some information on next steps and further research
that might be required or possible on the presented subject.




% Appendix chapters to be put here. They will be enumerated with capital letters
% if you  did not change the \documentclass options.
\appendix
% \include{content/appendix}

%Bibliography
% We strongly recommend to use bibtex to manage your bibliography. It helps you
% structure your references and helps avoiding missing important data for a correct
% quotation. If you have no other idea jabref (http://jabref.sourceforge.net/)
% might be a good idea (Java runtime environment needed).
\printbibliography

% Include the eidesstattliche Versicherung
% spellchecker:disable
%pagenumbering{null}

\

\cleardoublepage

\


\pagestyle{empty}

\selectlanguage{german}
\textbf{Versicherung an Eides Statt}\\

Ich versichere an Eides statt durch meine untenstehende Unterschrift,
\begin{itemize}
\item[-] dass ich die vorliegende Arbeit - mit Ausnahme der Anleitung durch die Betreuer - selbstständig ohne fremde Hilfe angefertigt habe und
\item[-] dass ich alle Stellen, die wörtlich oder annähernd wörtlich aus fremden Quellen entnommen sind, entsprechend als Zitate gekennzeichnet habe und
\item[-] dass ich ausschließlich die angegebenen Quellen (Literatur, Internetseiten, sonstige Hilfsmittel) verwendet habe und
\item[-] dass ich alle entsprechenden Angaben nach bestem Wissen und Gewissen vorgenommen habe, dass sie der Wahrheit entsprechen und dass ich nichts verschwiegen habe.
\end{itemize}
Mir ist bekannt, dass eine falsche Versicherung an Eides Statt nach \S 156 und \S 163 Abs. 1 des Strafgesetzbuches mit Freiheitsstrafe oder Geldstrafe bestraft wird.
\vfill
Duisburg, \today\\
$\overline{\parbox{4.8cm}{(Ort, Datum)}} ~~~~~~~~~~~~~~~~~~~~~~~~~~~ \overline{\parbox{7cm}{(Vorname Nachname)}}$


\end{document}
